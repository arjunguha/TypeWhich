\documentclass{article}
\usepackage[T1]{fontenc}
\usepackage{fullpage}
\usepackage{palatino}
\usepackage{xspace}
\usepackage{hyperref}
\usepackage{verbatim}
\usepackage{natbib}
\bibliographystyle{abbrvnat}
\setcitestyle{authoryear}

\newcommand{\system}{\textsc{TypeWhich}\xspace}
\newcommand{\kw}[1]{\textbf{\texttt{#1}}}
\newcommand{\metavar}[1]{\textit{#1}}

\title{\system Guide}
\author{Luna Phipps-Costin, Carolyn Jane Anderson, Michael Greenberg, and Arjun Guha}

\begin{document}

\maketitle

\tableofcontents

\section{Introduction}

\system is a type migration tool for the gradually-typed lambda calculus with
several extensions. Its distinguishing characteristics are the following:

\begin{enumerate}

\item \system formulates type migration as a MaxSMT problem.

\item \system always produces a migration, as long as the input program is
   well-scoped.

\item \system can optimize for different properties: it can produce the most
informative types, or types that ensure compatibility with un-migrated code.

\end{enumerate}

For more information on \system, see \citet{typewhich}.

This repository contains the source code for \system. In addition to the
core type migration algorithm, the \system executable has several auxiliary
features:

\begin{enumerate}

  \item It has a parser for the Grift programming language, which we use to
  infer types for the Grift benchmarks from \citet{kuhlenschmidt:grift};

  \item It has an interpreter for the GTLC, which we use in validation;
  
  \item It has an implementation of the gradual type inference algorithm
  from \citet{rastogi:gti}; and

  \item It includes a framework for evaluating type migration algorithms,
  which we use to compare \system to several algorithms from the
  literature~\cite{rastogi:gti,campora:migrating,migeed:decidable,siek:gti}.

\end{enumerate}
%
Finally, this repository contains several gradual typing benchmarks:
%
\begin{enumerate}

     \item The ``challenge set'' from \citet{typewhich};
     
     \item The benchmarks from \citet{migeed:decidable}; and
     
     \item The benchmarks from \citet{kuhlenschmidt:grift}.

\end{enumerate}

This document will guide you though building \system, using it on example
programs, and using the evaluation framework to reproduce our experimental
results.

\section{Building and Testing \system}

\system is built in Rust and uses Z3 under the hood. In principle, it should
work on macOS, Linux or Windows, though we have only tried it on macOS and
Linux. \emph{However}, our evaluation uses the implementation from
\citet{siek:gti}, which is an old piece of software that is difficult to build
on a modern platform. We have managed to compile it a Docker container and
produce a 32-bit Linux binary. It should be possible to build it for other
platforms, but it will require additional effort. Therefore, \textbf{we strongly
recommend using Linux to evaluate \system}.

\paragraph{Installing \system Dependencies}

To build \system from source, you will need:

\begin{enumerate}

\item The \href{https://rustup.rs/}{Rust language toolchain}.

\item The Z3 build dependencies and the ``usual'' build toolchain. On Ubuntu Linux, you can run the following
command to get them:

\begin{verbatim}
sudo apt-get install libz3-dev build-essential
\end{verbatim}

\item Python 3 and PyYAML to run the integration tests. These are installed by
default on most platforms. If you can run the following command successfully
then you already have them installed:
\begin{verbatim}
python3 -c "import yaml"
\end{verbatim}

\end{enumerate}

\paragraph{Installing Other Type Migration Tools}

\system does not require these dependencies, but they are necessary to reproduce
our evaluation.

\begin{enumerate}

\item \citet{migeed:decidable} is implemented in Haskell. We have written a
parser and printer for their tool that is compatible with \system. This
modified implementation is available at the following URL:

\url{https://github.com/arjunguha/migeed-palsberg-popl2020}

Build the tool as described in the repository, and then copy (or symlink) the
\texttt{MaxMigrate} program to \texttt{bin/MaxMigrate} in the \system
directory. On Linux, the executable is at:

\begin{verbatim}
migeed-palsberg-popl2020/.stack-work/install/x86_64-linux-tinfo6/
lts-13.25/8.6.5/bin/MaxMigrate
\end{verbatim}

\item \citet{siek:gti} is implemented in OCaml 3.12 (which is quite old).
The following repository has an implementation of the tool, with a modified
parser and printer that is compatible with \system:

\url{https://github.com/arjunguha/siek-vachharajani-dls2008}

Build the tool as described in the repository, and then copy (or symlink)
the \texttt{gtlc} program to \texttt{bin/gtubi} in the \system directory.

\textbf{Warning:} The repository builds a 32-bit Linux executable. You will
need to ensure that your Linux system has the libraries needed to run 32-bit
code.

\item \citet{campora:migrating} [FILL]
   
   \url{https://github.com/arjunguha/mgt}

\end{enumerate}

\paragraph{Building and Testing}

Use \texttt{cargo} to build \system:

\begin{verbatim}
cargo build
\end{verbatim}

Run the unit tests:
\begin{verbatim}
cargo test
\end{verbatim}
You may see a few ignored tests, but \emph{no tests should fail}.

Test \system using the Grift benchmarks:
\begin{verbatim}
./test-runner.sh grift grift
\end{verbatim}
\emph{No tests should fail.}

Finally, run the GTLC benchmarks without any third-party tools:
\begin{verbatim}
cargo run -- benchmark benchmarks.yaml \
  --ignore Gtubi MGT MaxMigrate > test.results.yaml
\end{verbatim}
You will see debugging output (on standard error), but the results will
be saved to the YAML file. Compare these results to known good results:
\begin{verbatim}
./bin/yamldiff test.expected.yaml test.results.yaml
\end{verbatim}
\emph{You should see no output, which indicates that there are no
differences.}

\section{Running \system}

The \system executable is symlinked to \texttt{bin/TypeWhich}. \system
expects its input program to be in a single file, and written in either
Grift (extension \texttt{.grift}) or in a superset of the 
gradually typed lambda calculus (extension \texttt{.gtlc}), shown in
Section~\ref{input-lang-gtlc}.

\paragraph{Example}
Create a file called \texttt{input.gtlc} with the following contents:

\begin{verbatim}
(fun f. (fun y. f) (f 5)) (fun x. 10 + x)
\end{verbatim}

This program omits all type annotations: \system assumes that omitted
annotations are all \kw{any}.

We can migrate the the program using \system in two modes:

\begin{enumerate}

\item In \emph{compatibility mode}, \system infers types but maintains
compatibility with un-migrated code:

\begin{verbatim}
$ ./bin/TypeWhich migrate input.gtlc
(fun f:any -> int. (fun y:int. f) (f 5)) (fun x:any. 10 + x)
\end{verbatim}

\item In \emph{precise mode}, \system infers the most precise type that it
can, though that may come at the expense of compatibility:

\begin{verbatim}
$ ./bin/TypeWhich migrate --precise inpuy.gtlc
(fun f:int -> int. (fun y:int. f) (f 5)) (fun x:int. 10 + x)
\end{verbatim}
     
\end{enumerate}

The \system executable supports several other sub-commands and flags. 
Run \texttt{./bin/TypeWhich --help} for more complete documentation.

\section{Input Language}\label{input-lang-gtlc}

\textbf{\url{./doc/doc.pdf} has the same content as this file, but with
slightly better formatting.}

\system supports a superset of the GTLC, written in the following syntax:

[FILL] A few cases missing

\begin{tabular}{rcll}
\metavar{b} & := & \kw{true} | \kw{false} & Boolean literal \\
\metavar{n} & := & ... | $-1$ | 0 | 1 | ... & Integer literals \\
\metavar{s} & := & \texttt{"..."} & String literals \\
\metavar{c} & := & b | n | s & Literals \\
\metavar{T} & := & \kw{any} & The unknown type \\
            & |  & \kw{int} & Integer type \\
            & |  & \kw{bool} & Boolean type \\
            & |  & \metavar{T}\textsubscript{1} \kw{->} \metavar{T}\textsubscript{2} & Function type \\
            & |  & \kw{(} \metavar{T} \kw{)} \\
\metavar{e} & := & \textit{x}  & Bound identifier \\
            & |  & \metavar{c} & Literal \\
            & |  & e \kw{:} T  & Type ascription \\
            & |  & \kw{(} \metavar{e} \kw{)} & Parenthesis \\
            & |  & \kw{fun} \metavar{x} \kw{.} \metavar{e} & Function \\
            & |  & \metavar{e}\textsubscript{1} \metavar{e}\textsubscript{2}
                 & Application \\
            & |  & \metavar{e}\textsubscript{1} \kw{+} \metavar{e}\textsubscript{2}
                 & Addition \\
            & |  & \metavar{e}\textsubscript{1} \kw{*} \metavar{e}\textsubscript{2}
                 & Multiplication \\
            & |  & \metavar{e}\textsubscript{1} \kw{=} \metavar{e}\textsubscript{2}
                 & Integer equality \\
            & |  & \metavar{e}\textsubscript{1} \kw{+?} \metavar{e}\textsubscript{2}
                 & Addition or string concatenation (overloaded) \\
            & |  & \kw{(}\metavar{e}\textsubscript{1}\kw{,}\metavar{e}\textsubscript{2}\kw{)}
                 & Pair \\
            & |  & \kw{fix} \metavar{f} \kw{.}\metavar{e}
                 & Fixpoint \\
            & |  & \kw{if} \metavar{e}\textsubscript{1} \kw{then} \metavar{e}\textsubscript{2} \kw{else} \metavar{e}\textsubscript{3}
                 & Conditional \\
            & |  & \kw{let} \metavar{x} \kw{=} \metavar{e}\textsubscript{1} \kw{in} \metavar{e}\textsubscript{2}
                 & Let binding \\
            & |  & \kw{let rec} \metavar{x} \kw{=} \metavar{e}\textsubscript{1} \kw{in} \metavar{e}\textsubscript{2}
                 & Recursive let binding \\

\end{tabular}

\section{Evaluation Framework}

\emph{To run the full suite of experiments, you will need to install the
third-party type migration tools.}

\system includes a framework for evaluating type migration algorithms, which is
driven by a \textsc{yaml} that specifies a list of type migration tools to
evaluate, and benchmark programs for the evaluation. The framework runs every
tool on every benchmark and then validates the result as follows:
\begin{enumerate}

\item It checks that the tool produces valid program, to verify that the tool
did not reject the program.

\item It runs the original program and the output of the tool and checks that
they produce the same result, to verify that the tool did not introduce a
runtime error.

\item In a gradually typed language, increasing type precision can make a
program incompatible with certain contexts. To check if this is the case, every
benchmark in the \textsc{yaml} file may be accompanied by a context that
witnesses the incompatibility: the framework runs the original and migrated
program in the context, to check if they produce different results.

\item The framework counts the number of \texttt{any}s that are eliminated
by the migration tool. Every eliminated \texttt{any} improves precision, 
\emph{may or may not} introduce an incompatibility, but this requires human
judgement. For example, in the
program \verb|fun x . x + 1|, annotating ``x'' with \texttt{int} does not
introduce an incompatibility. However, in \verb|fun x . x|, annotating ``x''
with \texttt{int} is an incompatibility. The framework flags these results
for manual verification. However, it allows the input \textsc{yaml} to specify
expected outputs to suppress these warnings when desired.

\end{enumerate}

The file \texttt{./benchmarks.yaml} drives the evaluation framework to
compare \system and several other type migration algorithms on a suite of
benchmarks. We recommend reading that file to understand benchmarking in
more depth.

To run the experiments, use the following command:
\begin{verbatim}
./bin/TypeWhich benchmark benchmarks.yaml > RESULTS.yaml
\end{verbatim}
It prints progress on standard error. The output is a YAML file of results.

\subsection{Validation}

\begin{enumerate}

\item The benchmarking script does a lot of validation itself.

\item In \texttt{RESULTS.yaml}, look for the string ``Disaster''. It should not
appear!

\item In \texttt{RESULTS.yaml}, look for the string \verb|manually_verify|.
These are results from experiments where (1) we could not crash the migrated
program, and (2) the migrated program has fewer `any`s than the original. So,
the table of results counts this migration as one that is 100\% compatible with
untyped contexts. But, it requires a manual check.

\item Finally, you can compare \texttt{RESULTS.yaml} with a known good output
from benchmarking:

\begin{verbatim}
./bin/yamldiff RESULTS.yaml expected.yaml
\end{verbatim}

\end{enumerate}

\subsection{Results}

To generate the summary table found in \citet{typewhich}, use the following
command:
\begin{verbatim}
./bin/TypeWhich latex-benchmark-summary RESULTS.yaml 
\end{verbatim}

To generate the appendix of results:

\begin{verbatim}
./bin/TypeWhich latex-benchmarks RESULTS.yaml 
\end{verbatim}

\section{Benchmarks}

The \system repository has several benchmarks:

\begin{enumerate}
   
\item The \texttt{migeed} directory contains the benchmarks
from Migeed et al., written in the concrete syntax of \system.

\item The \texttt{adversarial} directory contains the ``challenge set'' from
the \system paper.

\item The \texttt{grift-suite} directory contains tests from
\href{https://github.com/Gradual-Typing/Grift/tree/master/tests/suite}{Grift}.
The
\texttt{mu/} directory has been modified to use Dyn where it originally used recursive
types.

\item The \texttt{grift-suite/benchmarks} contains benchmarks from
\url{https://github.com/Gradual-Typing/benchmarks} with the following
adjustments:

\begin{enumerate}
\item The getters and setters in n-body have been removed. They were neither used
nor exported we opted to remove these functions from the benchmark. This is
discussed in the paper.
\item We have changed where in the program some benchmarks print a terminating
newline for consistency between the static and dynamic versions.
\item Benchmarks that rely on modules are removed
\end{enumerate}
\end{enumerate}

\bibliography{doc}

\end{document}