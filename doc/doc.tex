\documentclass{book}
\usepackage[T1]{fontenc}
\usepackage{fullpage}
\usepackage{palatino}
\usepackage{xspace}
\usepackage{hyperref}
\usepackage{verbatim}
\usepackage{natbib}
\bibliographystyle{abbrvnat}
\setcitestyle{authoryear}

\newcommand{\system}{\textsc{TypeWhich}\xspace}
\newcommand{\kw}[1]{\textbf{\texttt{#1}}}
\newcommand{\metavar}[1]{\textit{#1}}

\title{\system Guide}
\author{Luna Phipps-Costin, Carolyn Jane Anderson, Michael Greenberg, and Arjun Guha}

\begin{document}

\maketitle

\tableofcontents

\chapter{Introduction}

\system is a type migration tool for the gradually-typed lambda calculus with
several extensions. Its distinguishing characteristics are the following:

\begin{enumerate}

\item \system formulates type migration as a MaxSMT problem.

\item \system always produces a migration, as long as the input program is
   well-scoped.

\item \system can optimize for different properties: it can produce the most
informative types, or types that ensure compatibility with un-migrated code.

\end{enumerate}

Before you read the read of this guide or try to use \system, we strongly recommend
reading \citet{typewhich}, which describes \system in depth.

This repository contains the source code for \system. In addition to the
core type migration algorithm, the \system executable has several auxiliary
features:

\begin{enumerate}

  \item It has a parser for the Grift programming language, which we use to
  infer types for the Grift benchmarks from \citet{kuhlenschmidt:grift};

  \item It has an interpreter for the GTLC, which we use in validation;
  
  \item It has an implementation of the gradual type inference algorithm
  from \citet{rastogi:gti}; and

  \item It includes a framework for evaluating type migration algorithms,
  which we use to compare \system to several algorithms from the
  literature~\cite{rastogi:gti,campora:migrating,migeed:decidable,siek:gti}.

\end{enumerate}
%
Finally, this repository contains several gradual typing benchmarks:
%
\begin{enumerate}

     \item The ``challenge set'' from \citet{typewhich};
     
     \item The benchmarks from \citet{migeed:decidable}; and
     
     \item The benchmarks from \citet{kuhlenschmidt:grift}.

\end{enumerate}

This document will guide you though building \system, using it on example
programs, and using the evaluation framework to reproduce our experimental
results.

\section{Building and Testing \system}

\noindent
\textbf{For artifact evaluation, we strongly recommend using the \system
Virtual Machine and skipping this section.}

\paragraph{}
\system is built in Rust and uses Z3 under the hood. In principle, it should
work on macOS, Linux or Windows, though we have only tried it on macOS and
Linux. \emph{However}, our evaluation uses the implementation from
\citet{siek:gti}, which is an old piece of software that is difficult to build
on a modern platform. We have managed to compile it a Docker container and
produce a 32-bit Linux binary. It should be possible to build it for other
platforms, but it will require additional effort. Therefore, \textbf{we strongly
recommend using Linux to evaluate \system}.

\paragraph{Installing \system Dependencies}

To build \system from source, you will need:

\begin{enumerate}

\item The \href{https://rustup.rs/}{Rust language toolchain}.

\item The Z3 build dependencies and the ``usual'' build toolchain. On Ubuntu Linux, you can run the following
command to get them:

\begin{verbatim}
sudo apt-get install libz3-dev build-essential
\end{verbatim}

\item Python 3 and PyYAML to run the integration tests. These are installed by
default on most platforms. If you can run the following command successfully
then you already have them installed:
\begin{verbatim}
python3 -c "import yaml"
\end{verbatim}

\end{enumerate}

\paragraph{Installing Other Type Migration Tools}

\system does not require these dependencies, but they are necessary to reproduce
our evaluation.

\begin{enumerate}

\item \citet{migeed:decidable} is implemented in Haskell. We have written a
parser and printer for their tool that is compatible with \system. This
modified implementation is available at the following URL:

\url{https://github.com/arjunguha/migeed-palsberg-popl2020}

Build the tool as described in the repository, and then copy (or symlink) the
\texttt{MaxMigrate} program to \texttt{bin/MaxMigrate} in the \system
directory. On Linux, the executable is at:

\begin{verbatim}
migeed-palsberg-popl2020/.stack-work/install/x86_64-linux-tinfo6/
lts-13.25/8.6.5/bin/MaxMigrate
\end{verbatim}

\item \citet{siek:gti} is implemented in OCaml 3.12 (which is quite old).
The following repository has an implementation of the tool, with a modified
parser and printer that is compatible with \system:

\url{https://github.com/arjunguha/siek-vachharajani-dls2008}

Build the tool as described in the repository, and then copy (or symlink)
the \texttt{gtlc} program to \texttt{bin/gtubi} in the \system directory.

\textbf{Warning:} The repository builds a 32-bit Linux executable. You will
need to ensure that your Linux system has the libraries needed to run 32-bit
code.

\item \citet{campora:migrating} The following repository has our implementation
of the algorithm from \citet{campora:migrating}:
   
\url{https://github.com/arjunguha/mgt}

Build the tool as described in the repository and then copy (or symlink) the
the \texttt{mgt} program to \texttt{bin/mgt} in the \system directory.

\textbf{Note:} The original implementation by the authors of \citet{campora:migrating}
does not produce an ordinary migrated program as output. Instead, it produces a
BDD that can be interpreted as a family of programs. Our implementation of
their algorithm produces programs as output.

\end{enumerate}

\paragraph{Building and Testing}

Use \texttt{cargo} to build \system:

\begin{verbatim}
cargo build
\end{verbatim}

Run the unit tests:
\begin{verbatim}
cargo test
\end{verbatim}
You may see a few ignored tests, but \emph{no tests should fail}.

Test \system using the Grift benchmarks:
\begin{verbatim}
./test-runner.sh grift grift
\end{verbatim}
\emph{No tests should fail.}

Finally, run the GTLC benchmarks without any third-party tools:
\begin{verbatim}
cargo run -- benchmark benchmarks.yaml \
  --ignore Gtubi MGT MaxMigrate > test.results.yaml
\end{verbatim}
You will see debugging output (on standard error), but the results will
be saved to the YAML file. Compare these results to known good results:
\begin{verbatim}
./bin/yamldiff test.expected.yaml test.results.yaml
\end{verbatim}
\emph{You should see no output, which indicates that there are no
differences.}

Build \system in release mode (only needed for performance evaluation):

\begin{verbatim}
cargo build --release
\end{verbatim}

\chapter{Artifact Evaluation: Getting Started}
\label{getting-started} 

Before starting this chapter, we must either:

\begin{itemize}

     \item Use the \system Virtual Machine, or

     \item Install \system manually, along with all the third party
     tools we use for evaluation.

\end{itemize}

\noindent
\textbf{Warning:} \system uses the Z3 SMT solver under the hood, and different
versions of Z3 can produce slightly different results. The expected outputs that
we document in this guide were produced on the \system Virtual Machine.

\medskip

\noindent
If the following steps are successful, then we can be quite confident that
\system and all third-party tools are working as expected.

\begin{enumerate}
     
\item From a terminal window, enter the \system directory:

\begin{verbatim}
cd ~/typewhich
\end{verbatim}

\item Run the \system benchmarks and output results to \texttt{results.yaml}:

\begin{verbatim}
./bin/TypeWhich benchmark benchmarks.yaml > results.yaml
\end{verbatim}

This will take less than five minutes to complete. This command runs the GTLC
benchmarks using all five tools, including \system in two modes. Therefore, for
each benchmark, we will see six lines of output (on standard error). For example:
\begin{verbatim}
Running Gtubi on adversarial/01-farg-mismatch.gtlc ...
Running InsAndOuts on adversarial/01-farg-mismatch.gtlc ...
Running MGT on adversarial/01-farg-mismatch.gtlc ...
Running MaxMigrate on adversarial/01-farg-mismatch.gtlc ...
Running TypeWhich2 on adversarial/01-farg-mismatch.gtlc ...
Running TypeWhich on adversarial/01-farg-mismatch.gtlc ...
\end{verbatim}

There are three runs of third-party tools that take longer than 30 seconds, so you will \texttt{Killed} appear
three times. These are known shortcomings that are described in the paper.

\item Run the following command to ensure that the results are identical to
known good results:

\begin{verbatim}
./bin/yamldiff expected.yaml results.yaml
\end{verbatim}

There should be no output printed, which indicates that there are no
differences.

\item Run the following command to run \system on the Grift benchmarks:
\begin{verbatim}
./grift_inference.sh
\end{verbatim}

We expect to see \texttt{MATCHES} print several times, which indicates that
\system inferred exactly the same types that were written by the Grift authors
on that benchmark. However, we also expect to see a \texttt{Warning}, and
two mismatches on \texttt{n\_body} and \texttt{sieve}.

\end{enumerate}

At this point, we can investigate the artifact in more depth, which is the
subject of the next chapter.

\chapter{Artifact Evaluation: Step by Step Guide}

\noindent
\textbf{This chapter assumes you have completed the steps in Chapter~\ref{getting-started}.}

\section{Claims To Validate}

The paper makes the following claims that we validate in this chapter:
\begin{enumerate}

    \item Figure~15 summarizes the performance of several type migration tools
    on a a suite of benchmarks. This artifact generates that figure, and we can
    validate the data and benchmarking scripts in as much depth as desired.

    \item Section 6.5 runs \system on benchmarks written in Grift. These
    benchmarks have two versions: one that has no type annotations, and the other
    that has human-written type annotations. When run on the unannotated Grift
    benchmarks, \system recovers the human-written annotations on all but two
    of the Grift benchmarks. This artifact includes a script that produces this
    result.

    \item Section 6.6 reports that our full suite of benchmarks is 892 LOC, and
    \system{} takes three seconds to run on all of them. This artifact includes
    the script we use for the performance evaluation. It will take longer in a
    virtual machine, but should be roughly the same. i.e., it will be
    significantly less than 30 seconds.

\end{enumerate}

The rest of this section will is a step-by-step guide through repeating and
validating these claims.

\subsection{GTLC Benchmarks on Multiple Tools (Figure 15)}

In the previous chapter, we generated \texttt{results.yaml}. That
ran \system{} and all other tools on two suites of benchmarks:
\begin{enumerate}
        
\item All the benchmarks from \citet{migeed:decidable}, which are in the
\texttt{migeed} directory.
     
\item The ``challenge set''' from the paper, which are in the 
\texttt{adversarial} directory.

\end{enumerate}

The file \texttt{benchmarks.yaml} drives the benchmarking framework.
The top of the file lists the type migration tool, and is followed by
a list of benchmark files, and some additional information that needed to
produce results. The entire
benchmarking procedure is implemented in \texttt{src/benchmark.rs}, which
does performs the following steps on each benchmark:
\begin{enumerate}

\item It checks that the tool produces valid program, to verify that the tool
did not reject the program.

\item It runs the original program and the output of the tool and checks that
they produce the same result, to verify that the tool did not introduce a
runtime error.

\item In a gradually typed language, increasing type precision can make a
program incompatible with certain contexts. To check if this is the case, every
benchmark in the \textsc{yaml} file \emph{may} be accompanied by a context that
witnesses the incompatibility: the framework runs the original and migrated
program in the context, to check if they produce different results.

\item The framework counts the number of \texttt{any}s that are eliminated
by the migration tool. Every eliminated \texttt{any} improves precision, but
\emph{may or may not} introduce an incompatibility, but this requires human
judgement. For example, in the
program \verb|fun x . x + 1|, annotating ``x'' with \texttt{int} does not
introduce an incompatibility. However, in \verb|fun x . x|, annotating ``x''
with \texttt{int} is an incompatibility. The framework flags these results
for manual verification. However, it allows the input \textsc{yaml} to specify
expected outputs to suppress these warnings when desired.

\end{enumerate}

The file \texttt{results.yaml} is a copy of \texttt{benchmarks.yaml} with output
data added by the benchmarking framework. We use this file to generate Figure~15
in the paper. You should validate that table as follows:

\begin{enumerate}

\item Check that \texttt{results.yaml} does not have any errors: look for the
string ``Disaster'' in that file. It should not occur!

\item Regenerate the LaTeX snippet for the table with the following command:

\begin{verbatim}
./bin/TypeWhich latex-benchmark-summary results.yaml     
\end{verbatim}
The output that you will see is roughly the LaTeX code for Figure~15,
with two small differences:
\begin{enumerate}
\item It prints \texttt{TypeWhich2} instead of \texttt{TypeWhichC}, and
\item  \texttt{TypeWhich} instead of \texttt{TypeWhichP}.
\end{enumerate}

However, the order of rows and columns is exactly the same as the table in the
paper. It should be straightforward to check that the fractions in this output
are exactly the fractions reported in the table.

\end{enumerate}

\subsection{Grift Benchmarks with \system}

The Grift evaluation script (grift\_inference.sh) uses the --compare flag
of \system, which corresponds migrated types to the provided file's type
annotations and reports whether they match, ignoring annotations, coercions,
and unannotated identifiers.

On benchmarks for which it reports MATCHES, \system produced exactly the same type
annotations as the hand-typed versions.

On \texttt{n\_body}, verify that grift-suite/benchmarks/src/dyn/n\_body.grift
and \\
grift-suite/benchmarks/src/dyn/n\_body\_no\_unused\_funs.grift differ only
by the removal of unused getters and setters near the top of the
program. Note that \system's types on the adjusted benchmark with no unused
functions matches the hand-typed version.

On sieve, the warning refers to a lack of parsing support for recursive types.
As a result the mismatch message is less informative than inspection. To verify
exactly which types fail to migrate, run \system to migrate the types of the
program:

\begin{verbatim}
./bin/TypeWhich migrate grift-suite/benchmarks/src/dyn/sieve.grift
\end{verbatim}

Each bound identifier (excluding lets) will be printed, with its
type. Keeping that input open, manually inspect the hand-typed version at
grift-suite/benchmarks/src/static/sieve/single/sieve.grift. Consider, for
example, the first identifier, \texttt{stream-first}. The annotated program
declares stream-first to accept (Rec s (Tuple Int (-> s))) and return Int,
while \system's output accepts any and returns any. Inspecting each function
remaining, you will see that every (Rec s (Tuple Int (-> s))) is replaced
with the dynamic type. Also, some (but not all) integer types are migrated as
the dynamic type (because they hold values from projections out of tuples of
any). Note that the unit-terminated pair representation of tuples is visible in
stream-unfold, which otherwise has the expected type.

\subsection{Performance}

From the \system{} directory, run the following command:
\begin{verbatim}
time ./performance.sh
\end{verbatim}

The script will take roughly three seconds to complete. You can read the script
to verify that it runs \system{} on three suites of benchmarks:
\begin{enumerate}
     \item \texttt{migeed/*.gtlc}: the benchmarks from \citet{migeed:decidable},
     \item \texttt{adversarial/*.gtlc}: the ``challenge set'' from our paper, and
     \item \texttt{grift-suite/benchmarks/src/dyn/*.grift}: the benchmarks from \citet{kuhlenschmidt:grift}.
\end{enumerate}

\section{Exploring Type Migrations}

Our artifact includes several type migration tools, in addition to \system, and
we have hacked their parsers to work with the same concrete syntax, so that it
is easy to use any tool on the same program. We encourage you to try some out,
and to modify the benchmarks as well. Here are the available tools:

\begin{itemize}

\item To run \citet{migeed:decidable}:
     
\begin{verbatim}
./bin/MaxMigrate FILENAME.gtlc
\end{verbatim}

\item To run \citet{campora:migrating}:

\begin{verbatim}
./bin/mgt FILENAME.gtlc
\end{verbatim}

\item To run \citet{siek:gti}:

\begin{verbatim}
./bin/gtubi FILENAME.gtlc
\end{verbatim}

\item To run \citet{rastogi:gti}:
\begin{verbatim}
./bin/TypeWhich migrate --ins-and-outs FILENAME.gtlc
\end{verbatim}

\item To run \system{} and produce types that are safe in all contexts:
\begin{verbatim}
./bin/TypeWhich migrate FILENAME.gtlc
\end{verbatim}

\item To run \system{} and produce precise types that may not work in all
contexts:
\begin{verbatim}
./bin/TypeWhich migrate --precise FILENAME.gtlc
\end{verbatim}

\end{itemize}

\paragraph{Example}
Create a file called \texttt{input.gtlc} with the following contents:

\begin{verbatim}
(fun f. (fun y. f) (f 5)) (fun x. 10 + x)
\end{verbatim}

This program omits all type annotations: \system assumes that omitted
annotations are all \kw{any}.

We can migrate the the program using \system in two modes:

\begin{enumerate}

\item In \emph{compatibility mode}, \system infers types but maintains
compatibility with un-migrated code:

\begin{verbatim}
$ ./bin/TypeWhich migrate input.gtlc
(fun f:any -> int. (fun y:int. f) (f 5)) (fun x:any. 10 + x)
\end{verbatim}

\item In \emph{precise mode}, \system infers the most precise type that it
can, though that may come at the expense of compatibility:

\begin{verbatim}
$ ./bin/TypeWhich migrate --precise inpuy.gtlc
(fun f:int -> int. (fun y:int. f) (f 5)) (fun x:int. 10 + x)
\end{verbatim}
     
\end{enumerate}

\section{Input Language}\label{input-lang-gtlc}

\system supports a superset of the GTLC, written in the following syntax. Note
that the other tools do not support all the extensions documented below.

\begin{tabular}{rcll}
\metavar{b} & := & \kw{true} | \kw{false} & Boolean literal \\
\metavar{n} & := & ... | $-1$ | 0 | 1 | ... & Integer literals \\
\metavar{s} & := & \texttt{"..."} & String literals \\
\metavar{c} & := & b | n | s & Literals \\
\metavar{T} & := & \kw{any} & The unknown type \\
            & |  & \kw{int} & Integer type \\
            & |  & \kw{bool} & Boolean type \\
            & |  & \metavar{T}\textsubscript{1} \kw{->} \metavar{T}\textsubscript{2} & Function type \\
            & |  & \kw{(} \metavar{T} \kw{)} \\
\metavar{e} & := & \textit{x}  & Bound identifier \\
            & |  & \metavar{c} & Literal \\
            & |  & e \kw{:} T  & Type ascription \\
            & |  & \kw{(} \metavar{e} \kw{)} & Parenthesis \\
            & |  & \kw{fun} \metavar{x} \kw{.} \metavar{e} & Function \\
            & |  & \metavar{e}\textsubscript{1} \metavar{e}\textsubscript{2}
                 & Application \\
            & |  & \metavar{e}\textsubscript{1} \kw{+} \metavar{e}\textsubscript{2}
                 & Addition \\
            & |  & \metavar{e}\textsubscript{1} \kw{*} \metavar{e}\textsubscript{2}
                 & Multiplication \\
            & |  & \metavar{e}\textsubscript{1} \kw{=} \metavar{e}\textsubscript{2}
                 & Integer equality \\
            & |  & \metavar{e}\textsubscript{1} \kw{+?} \metavar{e}\textsubscript{2}
                 & Addition or string concatenation (overloaded) \\
            & |  & \kw{(}\metavar{e}\textsubscript{1}\kw{,}\metavar{e}\textsubscript{2}\kw{)}
                 & Pair \\
            & |  & \kw{fix} \metavar{f} \kw{.}\metavar{e}
                 & Fixpoint \\
            & |  & \kw{if} \metavar{e}\textsubscript{1} \kw{then} \metavar{e}\textsubscript{2} \kw{else} \metavar{e}\textsubscript{3}
                 & Conditional \\
            & |  & \kw{let} \metavar{x} \kw{=} \metavar{e}\textsubscript{1} \kw{in} \metavar{e}\textsubscript{2}
                 & Let binding \\
            & |  & \kw{let rec} \metavar{x} \kw{=} \metavar{e}\textsubscript{1} \kw{in} \metavar{e}\textsubscript{2}
                 & Recursive let binding \\

\end{tabular}

\chapter{Guide to Source Code}

The root TypeWhich directory includes a number of utilities, programs, and
source code (though most of \system is provided in src/):

\begin{enumerate}
    \item \texttt{adversarial/, grift-suite/benchmarks/, migeed/}: The three
    components of the \system benchmark suite, adversarial/ being original, and
    grift-suite/ and migeed/ adapted from the referenced research
    \item \texttt{doc/}: Source and render of this documentation
    \item \texttt{benchmarks.yaml}: This is the test harness configuration and data
    for the \system benchmarks framework, specifying to run the benchmarks and
    tools presented in the paper
    \item \texttt{expected.yaml}: Provides the expected behavior of the tool when
    configured with benchmarks.yaml
    \item \texttt{test.expected.yaml}: Provides the expected behavior of only
    \system / \citet{rastogi:gti} for testing the implementations
    \item \texttt{bin/}: Provides (and expects user to provide) symbolic links to tools
    \item \texttt{build.rs, Cargo.lock, Cargo.toml, target/}: Required build files for
    \system. Binaries are placed in target/
    \item \texttt{other-examples/}: Provides additional programs that are not
    interesting enough to be in the \system benchmark suite
    \item \texttt{grift\_inference.sh}: Evaluation tool for Grift benchmarks
    which compares if types produced are exactly the same as the static types
    provided in the suite
    \item \texttt{performance.sh, test-runner.sh, run\_tool.sh}: Tools that run
    \system on more programs or in release mode
    \item \textbf{\texttt{src/}}: The \system implementation, including implemention of
    \citet{rastogi:gti}
\end{enumerate}

Within \texttt{src/}, the following files are found:

\begin{enumerate}
    \item \textbf{\texttt{benchmark.rs}, \texttt{precision.rs}}: Provides the \system benchmarking
    framework
    \item \textbf{\texttt{cgen.rs}}: Generates the documented constraints of the \system
    algorithm and performs type migration
    \item \texttt{eval.rs}: An interpreter for the GTLC with explicit coercions
    \item \texttt{insert\_coercions.rs}: Type-directed coercion insertion for the GTLC,
    used for the interpreter. Not related to type migration
    \item \texttt{grift.l, grift.y, grift.rs, lexer.l, parser.y, pretty.rs}: Parsers and
    printers for Grift and the unified concrete syntax used by all tools
    \item \texttt{ins\_and\_outs/}: Our implementation of \citet{rastogi:gti}.
    \item \texttt{main.rs}: Entry point; options parsing
    \item \texttt{syntax.rs}: The language supported by \system. Also includes
    the --compare tool used for Grift evaluation
    \item \texttt{type\_check.rs}: Type-checking for programs with explicit coercions
    \item \texttt{z3\_state.rs}: Abstraction for the Z3 solver used for type inference
    in \system
\end{enumerate}

The core of the \system algorithm is found in cgen.rs. The constraints
specified in the paper are implemented in State::cgen (\textasciitilde{}line 52), with
comments resembling the notation from the paper. Of note are references to
\texttt{strengthen} and \texttt{weaken}, which are simply macros for
$(t1 = t2 \land w) \lor (t1 = * \land \texttt{ground(t2)} \land \neg w)$, w fresh; and
$(t1 = t2 \land w) \lor (t2 = * \land \texttt{ground(t1)} \land \neg w)$, w fresh
respectively. They are not to be confused with the \textsc{Weaken} function
from the paper.

State::negative\_any (\textasciitilde{}line 400) implements the \textsc{Weaken} algorithm from the paper.
typeinf\_options (\textasciitilde{}line 624) implements the \textsc{Migrate} algorithm in full.

\bibliography{doc}

\end{document}
